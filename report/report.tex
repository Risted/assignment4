%%%%% BEGIN Head
\documentclass[a4paper,10pt]{article}

\usepackage[utf8]{inputenc}
\usepackage[T1]{fontenc}
\usepackage[danish,english]{babel}
\usepackage{graphicx}
\usepackage[a4paper,margin=2.7cm]{geometry}

\usepackage[sc]{mathpazo} 
\usepackage{amsmath}
\usepackage{amssymb}
\usepackage{amsfonts}
\usepackage{enumerate}
\usepackage{longtable}

\usepackage{tikz}
\usetikzlibrary{arrows,decorations.pathmorphing,backgrounds,positioning,fit,matrix}

%%% Line breaks make som space between lines 
%   and begin all the way at the left,
%   out-comment for the standard LaTeX way.
\setlength{\parindent}{0pt}
\setlength{\parskip}{2ex} 


\renewcommand{\vec}[1]{\ensuremath {\mathbf #1}}

\renewcommand{\arraystretch}{1.2}

%%% Settings for the Instructor
\newcommand{\courseid}{DM510}
\newcommand{\coursename}{Operating systems}
\newcommand{\term}{Spring 2017}
\newcommand{\dept}{Department of Mathematics and Computer Science}

%%% Settings for the Student
%%%%%%%%%%%%%%%%%%%%%%%%%%%%%%%%%%%%%%%%%%%%%%%%%%%%%%%%%%%%
%%%%%%%%%%%%%%%%%%%%%%%%%%%%%%%%%%%%%%%%%%%%%%%%%%%%%%%%%%%%
\author{Jonas Andersen}
\makeatletter
\let\runauthor\@author
\let\runtitle\@title
\makeatother
%%%%%%%%%%%%%%%%%%%%%%%%%%%%%%%%%%%%%%%%%%%%%%%%%%%%%%%%%%%%
%%%%%%%%%%%%%%%%%%%%%%%%%%%%%%%%%%%%%%%%%%%%%%%%%%%%%%%%%%%%

%%%%%%%%%%%%%%%%%%%%%%%%%%%%%%%%%%%%%%%%%%%%%%%%%%%%%%%%%%%%

\usepackage{fancyhdr}
\usepackage{lastpage}
\usepackage{listings}
\lstset{language=C,
  basicstyle=\ttfamily\lst@ifdisplaystyle\footnotesize\fi,%\fontfamily{pzc}\selectfont,%
  stringstyle=\ttfamily,
  commentstyle=\ttfamily,
  showstringspaces=false,
  frame=lines, 
%  numbers=left,
  breaklines=true, tabsize=2,
  extendedchars=true,inputencoding=utf8
}
%\lstavoidwhitepre


\pagestyle{fancy} 
\lhead{{\sc \courseid -- \term }} 
\chead{}
%\rhead{CPR~nr.~\cprnr}
\cfoot{Page \thepage\ of \pageref{LastPage}}

\fancypagestyle{plain}{
\lhead{\dept\\
University of Southern Denmark, Odense}
\chead{}
\rhead{\today\\
\runauthor}
\lfoot{}
\rfoot{}
%\renewcommand{\headrulewidth}{0pt}
}


\title{\begin{flushleft}
\vspace{-4ex}
\courseid~-- \coursename \\[0.2cm]
{\Large Assignment 4: File System, \term \\[3ex]
\hrule}
\end{flushleft}
}


\date{}

%%%%% END Head


\begin{document}
\maketitle
Group: 

Jonas Andersen (jonan15)

Rikke Steinvig (riste15)

Emil Bløcher (emblo15)

\newpage
\tableofcontents

\newpage
%%%%%%%%%%%%%%%%%%%%%%%%%%%%%%%%%%%%%%%%%%%%%%%%%%%%%%%%%%%%
\section{Introduction}
%%%%%%%%%%%%%%%%%%%%%%%%%%%%%%%%%%%%%%%%%%%%%%%%%%%%%%%%%%%%


%%%%%%%%%%%%%%%%%%%%%%%%%%%%%%%%%%%%%%%%%%%%%%%%%%%%%%%%%%%%
\section{Design}
\label{design}
%%%%%%%%%%%%%%%%%%%%%%%%%%%%%%%%%%%%%%%%%%%%%%%%%%%%%%%%%%%%


%%%%%%%%%%%%%%%%%%%%%%%%%%%%%%%%%%%%%%%%%%%%%%%%%%%%%%%%%%%%
\section{Implementation}
%%%%%%%%%%%%%%%%%%%%%%%%%%%%%%%%%%%%%%%%%%%%%%%%%%%%%%%%%%%%


%\begin{lstlisting}
%struct dm510_buffer {
%	wait_queue_head_t readq, writeq;   // wait queues
%	char *buffer, *end;                // beginning of buffer, end of buffer
%	char *rp, *wp;                     // where to read, where to write
%	int buffersize;
%	int nreaders, nwriters;            // number of readers/writers.
%	int maxReaders, maxWriters;
%	struct semaphore sem;              // mutual exclusion semaphore
%};
%
%struct dm510_device {
%	struct cdev cdev;                  // Char device structure
%	struct dm510_buffer *read;         // Pointer to the buffer we want to read from
%	struct dm510_buffer *write;        // Pointer to the buffer we want to write to
%};
%\end{lstlisting}




%%%%%%%%%%%%%%%%%%%%%%%%%%%%%%%%%%%%%%%%%%%%%%%%%%%%%%%%%%%%
\section{Sleeping}
%%%%%%%%%%%%%%%%%%%%%%%%%%%%%%%%%%%%%%%%%%%%%%%%%%%%%%%%%%%%



%%%%%%%%%%%%%%%%%%%%%%%%%%%%%%%%%%%%%%%%%%%%%%%%%%%%%%%%%%%%
\section{Test}
%%%%%%%%%%%%%%%%%%%%%%%%%%%%%%%%%%%%%%%%%%%%%%%%%%%%%%%%%%%%


\begin{longtable}{|p{0.06\textwidth}|p{0.17\textwidth}|p{0.26\textwidth}|p{0.2\textwidth}|p{0.2\textwidth}|}
	\hline
	\textbf{Time-stamp} & \textbf{Command} & \textbf{Test} & \textbf{Expected output} & \textbf{Actual output} \endhead
	\hline
	0:30 & ./dm510\_load & Loads the module, running the init code and setting up the devices. & <Success message> & \textbf{Initialization done!!!} \\ 
	\hline
	0:34 & * \newline
	tests/testOpen & Opens the two devices in read/write mode. When the program terminates the devices are automatically released. & Positive numbers as file descriptors & \textbf{fd0: 3 \newline fd1: 4 \newline Released.\newline Released.} \\ 
	
	\hline
\end{longtable}


%%%%%%%%%%%%%%%%%%%%%%%%%%%%%%%%%%%%%%%%%%%%%%%%%%%%%%%%%%%%
\section{Discussion}
%%%%%%%%%%%%%%%%%%%%%%%%%%%%%%%%%%%%%%%%%%%%%%%%%%%%%%%%%%%%



%%%%%%%%%%%%%%%%%%%%%%%%%%%%%%%%%%%%%%%%%%%%%%%%%%%%%%%%%%%%
\section{Conclusion}
%%%%%%%%%%%%%%%%%%%%%%%%%%%%%%%%%%%%%%%%%%%%%%%%%%%%%%%%%%%%


\end{document}
